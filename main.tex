\define\field{\mathbb F}
\define\bits{\{0,1\}}

\definehead[exercise][subject]
\define\solution{\framed{Solution}}

\setuphead[exercise][
    textstyle=bold,
    after=\nowhitespace
]

\definestartstop[solution][
    before=\blank\startmarginrule,
    after=\stopmarginrule\blank
]

\starttext

\startstandardmakeup
\midaligned{Proofs, Arguments, and Zero-Knowledge}
\midaligned{Solutions by L. Russo}
\stopstandardmakeup

\starttitle[title=Preface]
This document contains some of my solutions to the excellent manuscript {\emph Proofs, Arguments, and Zero-Knowledge} by Justin Thaler.
That book is regularly updated (even on a daily basis) and it is not, as far as I know, versioned: so it becomes a bit hard to track whether or not some exercises are added, modified or removed.
For this reason, together with the solution, I have decided to attach the full text of the exercise as it is at the time of my reading.
\bigskip
I am always happy to receive feedback.
Also, no document is error-free: if you do find any, I would greatly appreciate it if you let me know; you can open an issue on GitHub for this.
\stoptitle

\starttitle[title=Solutions]
\exercise{Exercise 3.3}
Let $p=11$. Consider the function $f:\{0,1\}^2 \to \field_p$ given by $f(0,0) = 3$, $f(0,1) = 4$, $f(1,0) = 1$ and $f(1,1) = 2$. Write out an explicit expression for the multilinear extension $\tilde{f}$ of $f$.

\startsolution
We first write the Lagrange interpolation for $f$ as $\tilde{f}(x_1, x_2) = \sum_{w \in \bits^2} f(w)\Chi_w(x_1, x_2)$, where $\Chi_w(x_1, x_2) = \prod_{i=1}^2 (x_i w_i + (1-x_i)(1-w_i)$. We easily determine:
\startitemize[packed]
\item $\Chi_{00}(x_1, x_2) = (1-x_1)(1-x_2)$
\item $\Chi_{01}(x_1, x_2) = (1-x_1)x_2$
\item $\Chi_{10}(x_1, x_2) = x_1(1-x_2)$
\item $\Chi_{11}(x_1, x_2) = x_1x_2$
\stopitemize
\stopsolution
What is $\tilde{f}(2,4)$?
\startsolution
We only need to compute $\tilde{f}(x_1, x_2) = 3(1-x_1)(1-x_2) + 4(1-x_1)x_2 + x_1(1-x_2) +2x_1x_2$ in $(x_1, x_2) = (2, 4)$.
$\tilde{f}(2, 4) = 3(1-2)(1-4) + 4(1-2)4 + 2\cdot 4 +2\cdot 2\cdot 4 = 3$\stopsolution

Now  consider  the  function $f:\bits^3 \to \field_p$ given by $f(0,0,0) = 1$, $f(0,1,0) = 2$, $f(1,0,0) = 3$, $f(1,1,0) = 4$, $f(0,0,1) = 5$, $f(0,1,1) = 6$, $f(1,0,1) = 7$, $f(1,1,1) = 8$. What is $\tilde{f}(2,4,6)$?
\startsolution
$\tilde{f}(2,4,6) = 0$\footnote{Simply run multilinear_extension.py with the input values of the exercises. See also the next exercise.}
\stopsolution

\exercise{Exercise 3.4}
Fix some prime $p$ of your choosing. Write a Python program that takes as input an array of length $2^l$ specifying all evaluations of a function $f:\bits^l \to \field_p$ and a vector $r \in \field_p$, and outputs $\tilde{f}(r)$.
\startsolution
Check {\bf multilinear_extension.py} on my GitHub repository.
\stopsolution
\stoptitle
\stoptext
